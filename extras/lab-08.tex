\documentclass{tufte-handout}

\usepackage{xcolor}

% set image attributes:
\usepackage{graphicx}
\graphicspath{ {images/} }

% set hyperlink attributes
\hypersetup{colorlinks}

% ============================================================

% define the title
\title{SOC 4650/5650: Lab-08 - Accessing Demographic Data}
\author{Christopher Prener, Ph.D.}
\date{Spring 2019}
% ============================================================
\begin{document}
% ============================================================
\maketitle % generates the title
% ============================================================

\vspace{5mm}
\section{Directions}
Using data accessed from \texttt{tidycensus} and \texttt{tigris}, create \texttt{.csv} and \texttt{.shp} files describing SNAP benefit useage in Missouri and St. Louis County. Your entire project folder system, including data and RMarkdown output, should be uploaded to GitHub by \textbf{Wednesday, March 20\textsuperscript{th}} at 5:00pm.

\section{Analysis Development}
The goal of this section is to create a self contained project directory with all of the data, code, map documents, results, and documentation a project needs. Please ensure \textbf{all} required elements are present.

\vspace{5mm}
\section{Part 1: Download County Data}
The goal of this section is to be able to create a shapefile with SNAP benefit data for counties in Missouri and one for SNAP benefit data for census tracts in St. Louis.
\begin{enumerate}
\item Download the list of variables for the 2017 American Community Survey, and find the variable \texttt{B19058\_002} - what is its description?\sidenote{\textit{Hint:} Hover your mouse over the field if some of the text is obscured, and the full text will appear as a ``tooltip''.}
\item Download the data for variable \texttt{B19058\_002} for each county in Missouri.
\item Download a shapefile of all counties in Missouri that \textit{is not} generalized. 
\item Use data cleaning functions to limit and rename columns as necessary, and then join your tabular data to your geometric data before you export them as a shapefile.
\end{enumerate}

\vspace{5mm}
\section{Part 2: Download Census Tract Data}
The goal of this section is to be able to create a shapefile with SNAP benefit data for census tracts in St. Louis.
\begin{enumerate}
\setcounter{enumi}{4}
\item Download the data for variable \texttt{B19058\_002} for census tracts in St. Louis County (FIPS code is 189).
\item Download a shapefile of all census tracts in St. Louis County (FIPS code is 189) that \textit{is not} generalized. 
\item Use data cleaning functions to limit and rename columns as necessary, and then join your tabular data to your geometric data before you export them as a shapefile.
\end{enumerate}

% ============================================================
\end{document}